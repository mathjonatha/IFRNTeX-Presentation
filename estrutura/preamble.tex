\usepackage[T1]{fontenc}
\usepackage[brazil]{babel}
% \usepackage[utf8]{inputenc}

% PACOTES DE MATEMÁTICA
\usepackage{amsmath,amsthm}
% \usepackage{amssymb,amsfonts} % Desativar esse pacote quando estiver usando o mathdesign
\usepackage{xlop} % Pacote para fazer operações básicas de matemática
\opset{decimalsepsymbol={,}}
% \listfiles
\usepackage{mathtools}
\everymath{\displaystyle}
\usepackage{cancel} % pacote para inserir o sinal de corte em um número

% PACOTES tkz
\usepackage{tikz, tkz-euclide}
\usetikzlibrary{decorations.pathmorphing}

% PACOTES DE CORES
\usepackage{color, graphicx, xcolor, colortbl}

% PACOTES DE FORMATAÇÃO DE PÁGINA
\usepackage{multicol, fancyhdr}
\usepackage{lscape}
\usepackage{booktabs}
\usepackage{url}
\usepackage{caption}

% PACOTES DE NÚMERAÇÃO DE LISTAGEM
\usepackage{enumerate}

% OUTROS
\usepackage{lipsum}
\usepackage{latexsym, natbib, siunitx, setspace}
\usepackage{hyperref}% add hypertext capabilities

% Pacotes desativados que pode ser usado um dia
%\usepackage{esint} % Permite a utilização de novos símbolos de integrais.

% \setlength{\parindent}{0pt} % desligar as indentações

