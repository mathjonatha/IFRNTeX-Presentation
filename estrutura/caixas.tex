\usepackage{times}
\usepackage{eso-pic,graphicx} % Para funcionar o background


% CORES DO IF
\usepackage{xcolor}
\definecolor{corprimaria}{RGB}{10,48,123} 
\definecolor{corsecundaria}{RGB}{116,23,255}
\definecolor{corlinha}{RGB}{0,1,42}
\definecolor{corIFRN}{RGB}{50,160,65}

\definecolor{corexercicio}{RGB}{38,38,38}
\definecolor{cordefinicao}{RGB}{0,1,42}
% \definecolor{corexemplo}{RGB}{116,23,255}
\definecolor{corexemplo}{RGB}{38,38,38}

% CORES AnnWay
\definecolor{cor1}{RGB}{239,239,239}
\definecolor{cor2}{RGB}{205,25,30}
\definecolor{cor3}{RGB}{47,158,65}
\definecolor{cor4}{RGB}{38,38,38}
\definecolor{cor5}{RGB}{28,28,28}

% ESTILOS
\newtheoremstyle{geral}% Nome do estilo do teorema
{0pt}% Espaço acima
{0pt}% Espaço abaixo
{\normalfont}% Fonte do corpo
{}% Indent amount
{\small\bf\sffamily\color{cor4}}% Theorem head font
{:}% Punctuation after theorem head
{0.25em}% Space after theorem head
{} % Optional theorem note

% \newcounter{dummy}
% \newcounter{Exer}
% \newcounter{exer}
% \newcounter{exem}
% \theoremstyle{geral}
% \newtheorem{ExercicioT}[Exer]{Exercício resolvido}
% % \newtheorem{exercicioT}[exer]{ }
% \newtheorem{definicaoT}[dummy]{Conceito}
% \newtheorem{exemploT}[exem]{Exemplo}
% \newtheorem{obsT}{Observação}

\RequirePackage[framemethod=default]{mdframed} % Required for creating the theorem, definition, exercise and corollary boxes

% Caixa de exercícios
\newmdenv[skipabove=7pt,
skipbelow=7pt,
rightline=false,
leftline=true,
topline=false,
bottomline=false,
backgroundcolor=corexercicio!10,
linecolor=corexercicio,
innerleftmargin=5pt,
innerrightmargin=5pt,
innertopmargin=5pt,
innerbottommargin=5pt,
leftmargin=0cm,
rightmargin=0cm,
linewidth=4pt]{caixaEx}	

% Caixa de definição
\newmdenv[skipabove=7pt,
skipbelow=7pt,
rightline=false,
leftline=true,
topline=false,
bottomline=false,
backgroundcolor=corIFRN!10,
linecolor=corIFRN,
innerleftmargin=5pt,
innerrightmargin=5pt,
innertopmargin=5pt,
innerbottommargin=5pt,
leftmargin=0cm,
rightmargin=0cm,
linewidth=4pt]{caixaDE}	

% Caixa de exemplos
\newmdenv[skipabove=7pt,
skipbelow=7pt,
rightline=false,
leftline=true,
topline=false,
bottomline=false,
backgroundcolor=corexemplo!10,
linecolor=corexemplo,
innerleftmargin=5pt,
innerrightmargin=5pt,
innertopmargin=5pt,
innerbottommargin=5pt,
leftmargin=0cm,
rightmargin=0cm,
linewidth=4pt]{caixaExem}	

% Caixa de observações	  
\newmdenv[skipabove=7pt,
skipbelow=7pt,
rightline=false,
leftline=true,
topline=false,
bottomline=false,
backgroundcolor=cor2!10,
linecolor=cor2,
innerleftmargin=5pt,
innerrightmargin=5pt,
innertopmargin=5pt,
innerbottommargin=5pt,
leftmargin=0cm,
rightmargin=0cm,
linewidth=4pt]{caixaObs}


\newenvironment{Exercicio}{\begin{caixaEx}\begin{ExercicioT}}{\end{ExercicioT}\end{caixaEx}}
% \newenvironment{exercicio}{\begin{caixaEx}\begin{exercicioT}}{\end{exercicioT}\end{caixaEx}}
% \newenvironment{definicao}{\begin{caixaDE}\begin{definicaoT}}{\end{definicaoT}\end{caixaDE}}

\newenvironment{definicao}{\begin{caixaDE}}{\end{caixaDE}\vspace{.3cm}}
% \newenvironment{exemplo}{\begin{caixaExem} \begin{exemploT}}{\end{exemploT}\end{caixaExem}}

% \newenvironment{obs}{\begin{caixaObs}\begin{obsT}}{\end{obsT}\end{caixaObs}}

% \newenvironment{exercicio}[2][{\color{corexercicio}Exercício}]{\begin{trivlist}

% \item[\hskip \labelsep {\bfseries #1}\hskip \labelsep {\bfseries #2.}]}{\end{trivlist}}
