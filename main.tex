% Uppsala University presentation template
% Created 2021-03 by Mats Jonsson, mats@mekeriet.se

% Modelo de apresentação para o Instituto Federal do RN
% Modificado em 01/06/2021 por Matheus Jonatha, mthsjonatha@gmail.com

\documentclass[aspectratio=169,11pt]{if-beamer}
\usepackage{mwe} % for example image
\usepackage[T1]{fontenc}
\usepackage[brazil]{babel}

\graphicspath{{images/}, {style/graphics/}} % Put graphics and images in images/ directory

\usepackage{import}
    \usepackage[T1]{fontenc}
\usepackage[brazil]{babel}
% \usepackage[utf8]{inputenc}

% PACOTES DE MATEMÁTICA
\usepackage{amsmath,amsthm}
% \usepackage{amssymb,amsfonts} % Desativar esse pacote quando estiver usando o mathdesign
\usepackage{xlop} % Pacote para fazer operações básicas de matemática
\opset{decimalsepsymbol={,}}
% \listfiles
\usepackage{mathtools}
\everymath{\displaystyle}
\usepackage{cancel} % pacote para inserir o sinal de corte em um número

% PACOTES tkz
\usepackage{tikz, tkz-euclide}
\usetikzlibrary{decorations.pathmorphing}

% PACOTES DE CORES
\usepackage{graphicx, xcolor, colortbl}

% PACOTES DE FORMATAÇÃO DE PÁGINA
\usepackage{multicol, fancyhdr}
\usepackage{lscape}
\usepackage{booktabs}
\usepackage{url}
\usepackage{caption}

% PACOTES DE NÚMERAÇÃO DE LISTAGEM
\usepackage{enumerate}
% \usepackage{enumitem}


% OUTROS
\usepackage{lipsum}
\usepackage{latexsym, natbib, siunitx, setspace}
\usepackage{hyperref}% add hypertext capabilities

% Pacotes desativados que pode ser usado um dia
%\usepackage{esint} % Permite a utilização de novos símbolos de integrais.

% \setlength{\parindent}{0pt} % desligar as indentações

    \usepackage{times}
\usepackage{eso-pic,graphicx} % Para funcionar o background


% CORES DO IF
\usepackage{xcolor}
\definecolor{corprimaria}{RGB}{10,48,123} 
\definecolor{corsecundaria}{RGB}{116,23,255}
\definecolor{corlinha}{RGB}{0,1,42}
\definecolor{corIFRN}{RGB}{50,160,65}

\definecolor{corexercicio}{RGB}{38,38,38}
\definecolor{cordefinicao}{RGB}{0,1,42}
% \definecolor{corexemplo}{RGB}{116,23,255}
\definecolor{corexemplo}{RGB}{38,38,38}

% CORES AnnWay
\definecolor{cor1}{RGB}{239,239,239}
\definecolor{cor2}{RGB}{205,25,30}
\definecolor{cor3}{RGB}{47,158,65}
\definecolor{cor4}{RGB}{38,38,38}
\definecolor{cor5}{RGB}{28,28,28}

% ESTILOS
\newtheoremstyle{geral}% Nome do estilo do teorema
{0pt}% Espaço acima
{0pt}% Espaço abaixo
{\normalfont}% Fonte do corpo
{}% Indent amount
{\small\bf\sffamily\color{cor4}}% Theorem head font
{:}% Punctuation after theorem head
{0.25em}% Space after theorem head
{} % Optional theorem note

% \newcounter{dummy}
% \newcounter{Exer}
% \newcounter{exer}
% \newcounter{exem}
% \theoremstyle{geral}
% \newtheorem{ExercicioT}[Exer]{Exercício resolvido}
% % \newtheorem{exercicioT}[exer]{ }
% \newtheorem{definicaoT}[dummy]{Conceito}
% \newtheorem{exemploT}[exem]{Exemplo}
% \newtheorem{obsT}{Observação}

\RequirePackage[framemethod=default]{mdframed} % Required for creating the theorem, definition, exercise and corollary boxes

% Caixa de exercícios
\newmdenv[skipabove=7pt,
skipbelow=7pt,
rightline=false,
leftline=true,
topline=false,
bottomline=false,
backgroundcolor=corexercicio!10,
linecolor=corexercicio,
innerleftmargin=5pt,
innerrightmargin=5pt,
innertopmargin=5pt,
innerbottommargin=5pt,
leftmargin=0cm,
rightmargin=0cm,
linewidth=4pt]{caixaEx}	

% Caixa de definição
\newmdenv[skipabove=7pt,
skipbelow=7pt,
rightline=false,
leftline=true,
topline=false,
bottomline=false,
backgroundcolor=corIFRN!10,
linecolor=corIFRN,
innerleftmargin=5pt,
innerrightmargin=5pt,
innertopmargin=5pt,
innerbottommargin=5pt,
leftmargin=0cm,
rightmargin=0cm,
linewidth=4pt]{caixaDE}	

% Caixa de exemplos
\newmdenv[skipabove=7pt,
skipbelow=7pt,
rightline=false,
leftline=true,
topline=false,
bottomline=false,
backgroundcolor=corexemplo!10,
linecolor=corexemplo,
innerleftmargin=5pt,
innerrightmargin=5pt,
innertopmargin=5pt,
innerbottommargin=5pt,
leftmargin=0cm,
rightmargin=0cm,
linewidth=4pt]{caixaExem}	

% Caixa de observações	  
\newmdenv[skipabove=7pt,
skipbelow=7pt,
rightline=false,
leftline=true,
topline=false,
bottomline=false,
backgroundcolor=cor2!10,
linecolor=cor2,
innerleftmargin=5pt,
innerrightmargin=5pt,
innertopmargin=5pt,
innerbottommargin=5pt,
leftmargin=0cm,
rightmargin=0cm,
linewidth=4pt]{caixaObs}


\newenvironment{Exercicio}{\begin{caixaEx}\begin{ExercicioT}}{\end{ExercicioT}\end{caixaEx}}
% \newenvironment{exercicio}{\begin{caixaEx}\begin{exercicioT}}{\end{exercicioT}\end{caixaEx}}
% \newenvironment{definicao}{\begin{caixaDE}\begin{definicaoT}}{\end{definicaoT}\end{caixaDE}}

\newenvironment{definicao}{\begin{caixaDE}}{\end{caixaDE}\vspace{.3cm}}
% \newenvironment{exemplo}{\begin{caixaExem} \begin{exemploT}}{\end{exemploT}\end{caixaExem}}

% \newenvironment{obs}{\begin{caixaObs}\begin{obsT}}{\end{obsT}\end{caixaObs}}

% \newenvironment{exercicio}[2][{\color{corexercicio}Exercício}]{\begin{trivlist}

% \item[\hskip \labelsep {\bfseries #1}\hskip \labelsep {\bfseries #2.}]}{\end{trivlist}}

    % QUESTÕES
% configurações das questões, bem como: pontuação e estrutura.

\usepackage{tasks} % cria lista curta
% \usepackage{exsheets} % cria questoes
% \SetupExSheets[points]{ name=ponto/s,number-format=\color{blue}} % define as configurações de pontuação das questões, e a cor da pontuação.

% \DeclareInstance{exsheets-heading}{fancy-wp}{default}{
% toc-reversed = true ,
% indent-first = true ,
% vscale = 2 ,
% pre-code = \rule{\linewidth}{1pt} ,
% post-code = \rule{\linewidth}{1pt} ,
% title-format = \large\scshape\color{rgb:red,0.65;green,0.04;blue,0.07} ,
% number-format = \large\bfseries\color{rgb:red,0.02;green,0.04;blue,0.48} ,
% points-format = \itshape ,
% points-pre-code = ( ,
% points-post-code = ) ,
% join =
% {
% number[r,B]title[l,B](.333em,0pt) ;
% number[r,B]points[l,B](.333em,0pt)
% } ,
% attach = { main[hc,vc]number[hc,vc](0pt,0pt) }
% }

% %\SetupExSheets{headings=fancy-wp} % estilo diferente para o topo do enunciado com o nome " Exercício
    
% ------------------- Title -------------------------
\title[Matheus Jonatha]{TÍTULO DE TESTE}
\subtitle{Suas vantagens e desvantagens}
\author[\today]{Matheus JonathaS}
\date{2021-12-24}
\institute[IFRN]{Instituto Federal do RN}

\titulo{Licenciatura em Matemática}{Professor: Matheus Jonatha}

\begin{document}


% Logo page
\logopage

% Main presentation title page
\titlepage

\begin{frame}
    \tableofcontents
\end{frame}

% Title, image and subtitle
\titleimage{Title and graphics}{Description of picture.}{example-image}

% Title page
\anothertitle{Another title page}{and another subtitle}

% Two-column with image 
\begin{frame}
    \Frametitle{Two-column layout with graphics}
    Normal text.

    \begin{columns}
        \column{0.5\textwidth}
        \begin{itemize}
        \item Bullet
        \item Point
        \item List
        \end{itemize}
        \begin{enumerate}
        \item Bullet
        \item Point
        \item List
        \end{enumerate}
        

        \column{0.5\textwidth}
        \includegraphics[keepaspectratio,width=\textwidth,height=\textheight]{example-image}
      \end{columns}
    
\end{frame}

% Full-size image
\fullimage{example-image}


\section{Potenciação}
    \subsection{Ideias iniciais}
    
    \begin{frame}
        \Frametitle{Ideias iniciais}
        \begin{definicao}
           Representamos por \(a^n\), a potência de \textbf{base} real \(a\) e \textbf{expoente} inteiro \(n\).
        \end{definicao}
           Definimos assim a potência \(a^n\) nos casos a seguir:
        
        \vspace{.3cm}
        % \begin{itemize}
            \textbf{Caso 1:} \textit{Expoente inteiro maior que \(1\).}
            \vspace{.1cm} 
            \\
                Potência de expoente inteiro maior que \(1\) é o \textbf{produto de tantos fatores iguais à base} quantas forem as \textbf{unidades do expoente}.
                
                Assim:
                        \[ a^n = a \cdot a \cdot a \cdots a \cdot a\]
\pause     
                \textbf{Exemplos}
                    \begin{tasks}(2)
                    \task \( 5^4 = 5 \cdot 5 \cdot 5 \cdot 5 = 625 \)
                    \task \( -5^4 = -(5 \cdot 5 \cdot 5 \cdot 5) = -625 \)
                    \task \( (-5)^2 = (-5) \cdot (-5) \cdot (-5) \cdot (-5)=\)
                    \task \( 1^6 = 1 \cdot 1 \cdot 1 \cdot 1 \cdot 1 \cdot 1 = 1\)
                    \task \( \left( \frac{-2}{3} \right)^2 = \left( -\frac{2}{3} \right) \cdot \left( -\frac{2}{3} \right) = \frac{4}{9}\)
                    \task \( \left( \frac{3}{\sqrt{3}} \right)^2 = \left( \frac{3}{\sqrt{3}} \right) \cdot \left( \frac{3}{\sqrt{3}} \right) = 3 \)
                \end{tasks}
        % \end{itemize}
    \end{frame}
    \subsection{Potência de expoente 1}
    \begin{frame}{Potência de expoente 1}{Potenciação}
        \textbf{Caso 2:} \textit{Expoente \(1\).}
            \vspace{.1cm} 
            \\
            Toda potência de expoente \(1\) é igual à base.
            
            Assim:
                \[ a^1 = a \]
            \textbf{Exemplos:}
                \begin{tasks}(2)
                    \task \(5^1 = 5\)
                    \task \( - \left( \frac{3}{2} \right)^1 = - \left( \frac{3}{2} \right)\)
                    \task \(\left( \sqrt[3]{49} \right)^1 = \left( \sqrt[3]{49} \right)\)
                    \task \(\left( \frac{\sqrt{3} + 2}{3} \right)^1 = \left( \frac{\sqrt{3} + 2}{3} \right)\)
                \end{tasks}
    \end{frame}
\subsection{Potência de expoente zero}
    \begin{frame}{Potência de expoente zero}{Potenciação}
        \textbf{Caso 3:} \textit{Expoente zero.}
        
        
            Toda potência de expoente zero é igual a 1. Para qualquer base \(a \neq 0\).
            
            Assim:
                \[ a^0 = 1 \]
            \textbf{Exemplos:}
                \begin{tasks}(2)
                    \task \( 168^0 = 1 \)
                    \task \( - \left( \frac{3}{5} \right)^0 = -1 \)
                    \task \( \left( \sqrt{625} \right)^0 = 1 \)
                    \task \( (-25)^0 = 1 \)
                \end{tasks}
    \end{frame}
\subsection{Potência de expoente negativo}
    \begin{frame}{Potência de expoente real negativo}{Potenciação}
        \textbf{Caso 3:} \textit{Expoente negativo.}
            \vspace{.1cm} 
            \\
            Toda potência de expoente inteiro negativo e base não-nula é igual à potência de base igual ao inverso da base dada e expoente igual ao oposto do expoente dado.
            
            Assim:
                \[ a^{-n} = \left( \frac{1}{a} \right)^n \]
            \textbf{Exemplos:}
                \begin{tasks}(2)
                    \task \( \left( \frac{7}{5} \right)^{-3} = \left( \frac{5}{7} \right)^{3} = \frac{125}{343}\)
                    \task \( \left( -3 \right)^{-4} = \left( \frac{-1}{3} \right)^4 = \frac{1}{81} \)
                \end{tasks}
                \vspace{.3cm}
\pause
            \textbf{Observações:}
                \begin{multicols}{2}
                \begin{itemize}
                    \item \(a = 0\) e \( n > 0 \Rightarrow a^n = 0\)
                    \item \(a = 0\) e \( n < 0 \Rightarrow \nexists~ a^n \in \mathbb{R}\)
                    \item \(a > 0 \Rightarrow a^n > 0\)
                    \item \( a < 0\) e \(n\) par \(\Rightarrow a^n > 0\)
                    \item \(a < 0\) e \(n\) ímpar \(\Rightarrow a^n < 0\)
                \end{itemize}
                \end{multicols}
    \end{frame}
    
    \begin{frame}{Produto de potências de mesma base}{Propriedades}
        Considera os números reais \(a,b\), e os números inteiros \(m\) e \(n\). Temos que:
        
        \[a^n \cdot a^m = a^{n + m}\]
        
        Para multiplicarmos potências de mesma base, \textbf{conservamos a base} e \textbf{adicionamos os expoentes}.
        \vspace{.5cm}
        
        \textbf{Exemplos:}
        
        \begin{tasks}(2)
            \task \(2^2 \cdot 2^3 = 2^{2+3} = 2^5\)
            \task \(3^2 \cdot 3^2 \cdot 3^6 = 2^{2+2+3} = 2^7\)
            \task \(5^x \cdot 5^{-x+1} = 5^{x + (-x+1)} = 5^1\)
            \task \(7^{-5} \cdot 7^{3} = 7^{-5+3} = 7^{-2}\)
        \end{tasks}
        
    \end{frame}
    
    \begin{frame}{Quociente de potências de mesma base}{Propriedades}
        Para dividirmos potências de mesma base, \textbf{conservamos} a base e \textbf{subtraímos os expoentes}.
        
        \[\frac{a^n}{a^m} = a^{n-m}, ~(a \neq 0)\]
        
        \vspace{1cm}
        \textbf{Exemplos:}
        
        \begin{tasks}(2)
            \task \(\frac{2^3}{2^2} = 2^{3-2} = 2^1\)
            \task \(\frac{5^{-3}}{5^{2}} = 5^{-3+2} = 5^{-1}\)
            \task \(\frac{4^x}{4^3} = 4^{x-3}\)
            \task \(\frac{4^{x+2}}{4^{x-3}} = 4^{(x+2)-(x-3)} = 4^{5}\)
        \end{tasks}
    \end{frame}
    
    \begin{frame}{Produto de potências de mesmo expoente}{Propriedades}
        \[a^n \cdot b^n = (a \cdot b)^n\]
        Para multiplicarmos potências de mesmo expoente, \textbf{conservamos o expoente} e \textbf{multiplicamos as bases}.
        
        \vspace{1cm}
        \textbf{Exemplos:}
        
        \begin{tasks}(2)
            \task \(2^3 \cdot 6^3 = (2 \cdot 6)^3 = 12^3\)
            \task \(5^5 \cdot 9^5 = (5 \cdot 9)^5 = 45^5\)
            \task \(3^2 \cdot 5^2 \cdot 4^2 \cdot 7^2= (3 \cdot 5 \cdot 4 \cdot 7)^2 = 420^2\)
            \task \(x^4 \cdot y^4 \cdot z^4 = (x \cdot y \cdot z)^4\)
        \end{tasks}
    \end{frame}
    
    \begin{frame}{Quociente de potências de mesmo expoente}{Propriedades}
        \[\frac{a^n}{b^n} = \left( \frac{a}{b} \right)^n,~(b \neq 0) \]
        Para dividirmos potências de mesmo expoente, \textbf{conservamos o expoente} e \textbf{dividimos as bases}.
        
        
        \vspace{1cm}
        \textbf{Exemplos:}
        \begin{tasks}(2)
            \task \(\frac{3^2}{2^2} = \left(\frac{3}{2} \right)^2\)
            \task \(\frac{4^{-2}}{3^{-2}} = \left( \frac{4}{3}\right)^{-2}\)
            \task \(\frac{x^7 \cdot y^7}{z^7} = \left(\frac{x \cdot y}{z} \right)^7\)
            \task \(\frac{1}{5^4} = \left( \frac{1}{5}\right)^4\)
        \end{tasks}
    \end{frame}
    
    \begin{frame}{Potência de uma potência}{Propriedades}
        \[(a^n)^m = a^{n \cdot m}\]
        pa elevarmos uma potência a um nov expoente, \textbf{conservamos a base} e \textbf{multiplicamos os expoentes}.
        \vspace{1cm}
        \textbf{Exemplos:}
        \begin{tasks}(2)
            \task \((3^3)^2 = 3^{(3 \cdot 2)} = 3^6\)
            \task \((x^4)^3 = x^{12}\)
            \task \(\left( \left( 4^2 \right)^3 \right)^4 = 4^{(2 \cdot 3 \cdot 4)} = 4^{24}\)
            \task \((x^6)^3 = (x^3)^6 = x^{18}\)
        \end{tasks}
    \end{frame}
    % \begin{frame}{Resumo das propriedades}{Potência}
    %     \begin{itemize}
    %         \item \(a^n \cdot a^m = a^{n + m}\)
    %         \item \( \frac{a^n}{a^m} = a^{n-m} \)
    %         \item \( a^n \cdot b^n = (a \cdot b)^n \)
    %         \item \( \frac{a^n}{b^n} = \left( \frac{a}{b} \right)^n\)
    %         \item \( (a^n)^m = a^{n \cdot m} \)
    %     \end{itemize}
    % \end{frame}
    
    
    \begin{frame}{Título}{subtitulo}
        Aqui vai o corpo do texto
        
    \end{frame}
\end{document}

